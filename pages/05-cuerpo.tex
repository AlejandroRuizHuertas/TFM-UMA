\section{Introducción}

\subsection{Motivación}

Cita de artículo \cite{greenwade93}

Código de ejemplo:
\begin{lstlisting}[language=Python]
def main():
  print("Cita de articulo \cite{greenwade93}")
\end{lstlisting}

O, alternativamente, para escribir {\it pseudo}-código:
\begin{algorithm}[H]
  \caption{Ejemplo}\label{euclid}
  \begin{algorithmic}[1]
    \Procedure{Euclid}{$a,b$} \Comment{The g.c.d. of a and b}
        \State $r\gets a \bmod b$
        \While{$r\not=0$} \Comment{We have the answer if r is 0}
            \State $a \gets b$
            \State $b \gets r$
            \State $r \gets a \bmod b$
        \EndWhile\label{euclidendwhile}
        \State \textbf{return} $b$\Comment{The gcd is b}
    \EndProcedure
  \end{algorithmic}
\end{algorithm}

\tikzstyle{box} = [draw, rectangle, minimum height=3em, minimum width=3em]
\tikzstyle{cir} = [draw, circle, line width=0.5pt]
\tikzstyle{container} = [draw, rectangle, dashed, inner sep=1em]
\begin{figure*}[h]
  \centering
  \begin{tikzpicture}[auto, node distance=5cm]
      % Place nodes
      \node [box, fill={rgb:red,1;green,2;blue,5}] (node1) {Foo};
      \node [container, fit=(node1), label=north:{main}] (node1container) {};
      \node [cir, right=of node1] (node2) {Bar};
      % Connect nodes
      \draw [->, rounded corners=0.2cm] (node1) -- node {link} (node2);
  \end{tikzpicture}
  \caption{Diagrama de ejemplo.}
\end{figure*}

\subsection{Estado del arte}

\cleardoublepage

\section{Diseño}
\subsection{Implementación}

\cleardoublepage

\section{Resultados}
\subsection{Conclusiones}
\subsection{Futuras líneas de trabajo}

\cleardoublepage

\section{Bibliografía}
\printbibliography[heading=none]

\cleardoublepage

\section{Anexos}